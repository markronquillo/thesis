\section{Methodology}
This section states how the EMS-GT2 and the proposed speedup techniques was implemented and was also evaluated using synthetic datasets for some challenge $(l, d)$ instances. A study \cite{pms2015} proposed that $(l, d)$ instances where $d$ is the largest integer value for which the expected number of motifs of length $l$ would occur in the input by random chance and does not exceed a constant value (500) are categorized as challenging instances. The following $(l, d)$-challenge instances $(9, 2)$, $(11, 3)$, $(13, 4)$, $(15, 5)$ and $(17, 6)$ were used for the evaluation of the algorithms.

\subsection{Datasets}
Algorithms that solve PMS \cite{pevzner2000combinatorial, pms2014, pms2015} use a dataset containing $20$ string sequences where each nucleotide is in $\Sigma = \{a, c, g, t\}$. Each string sequence is $600$ base pairs (bp) long and each nucleotide is randomly generated with equal chance of being selected. A motif is then generated and for each string sequence in the dataset, a $d$-neighbor is planted at a random position. In this study, a generator that outputs dataset with this configuration was used to evaluate the algorithms. Furthermore, a converter program was used to translate dataset into FASTA format in order to execute qPMS9.


\subsection{Implementation}
The EMS-GT2 maintains a $4^l$ motif search space and is represented by array of bits for space efficiency. Previous speedup technique exploit this manner of representing the search space by generating the neighborhood of an $l$-mer by blocks instead of per bit. In this study, we took advantage of this block-processing approach in testing of candidate motifs. The Test phase checks if a candidate motif $c$ is in the remaining $n - n'$ sequences by comparing if there is at least one $l$-mer in each sequence that is within $d$-distance from $c$. If a candidate motif $x$ is eliminated and a candidate motif $y$ is within $k$-block from $x$, it is possible to filter out some $l$-mers in the sequence where $x$ got eliminated when we test $y$.

\subsection{Parameter Fine Tuning}
The EMS-GT2 defines an integer value $n'$ $(1 < n' < n)$ that divides the dataset into two set of sequences. The first $n'$ sequences will be used in the Generate phase while the remaining will be assigned in the Test phase. Previous experimentations \cite{journal} showed that it is efficient for the algorithm to set the value of $n'$ to $10$. Technically, $n'$ dictates how big is the size of the set of candidate motifs $\mathcal{C}$ to be evaluated if they are in the remaining $n - n'$ sequences. Inline with this, we run an experimentation that records the average runtime of the algorithm with the speedup techniques over 5 tests and having different values for $n'$. The values for $n'$ ranges from 5 to 10 in this experiment, since we only want to make the candidate motif set large enough for our speedup technique to take effect.

% include mismatch nprime table
\begin{table}[h] %Empty Blocks
	\renewcommand{\arraystretch}{1.3}
	\label{tbl:nprime-speedup}
	\centering
	\begin{tabular}{|c|c|c|c|c|c|}
		\hline 
		\bfseries\boldmath $Sequence$ & 
		\bfseries\boldmath $(9,2)$ & 
		\bfseries\boldmath $(11,3)$ & 
		\bfseries\boldmath $(13,4)$ & 
		\bfseries\boldmath $(15,5)$ & 
		\bfseries\boldmath $(17,6)$ \\
		\hline
			5	& 	0.07 s			& 	0.42 s 			&	2.37 s			&	17.86 s				&	148.62 s\\
			6	& 	0.06 s			& 	0.29 s			& 	1.54 s			&	12.64 s				&	116.01 s\\
			7	& 	0.05 s			& 	0.21 s			& 	1.01 s			&	10.70 s				&	\textbf{111.94 s}\\
			8	& 	0.05 s			&	0.18 s			& 	0.84 s			&	\textbf{10.38 s}	&	119.82 s\\
			9	& 	0.03 s 			& 	0.14 s			& 	\textbf{0.75 s}	&	11.01 s				&	132.10 s\\
			10	& 	\textbf{0.03 s} &	\textbf{0.13 s}	& 	0.76 s			&	11.90 s				&	146.10 s\\
		\hline\end{tabular}
	\caption{Runtime evaluation of the algorithm with speedup over different $n'$ values.}
\end{table}

Table \ref{tbl:nprime-speedup} shows the ideal $n'$ value for $(l, d)$-challenge instances. For $(l, d)$ instances where $l \leq 11$, the ideal value for $n'$ is still 10. This is true due to the efficiency of the Generate phase in small instances of the problem. For $(13, 4)$, $(15, 5)$ and $(17, 5)$, different $n'$ were used in the evaluation which are 9, 8 and 7 respectively.


\subsection{Evaluation}

For evaluation of the algorithms, we compared the EMS-GT2 to EMS-GT and the current state-of-the-art algorithm qPMS9. We used the challenging $(l, d)$ instances defined in \cite{pms2007, pms2015}. The instances used in the evaluation are the following: $(9,2)$, $(11, 3)$, $(13, 4)$, $(15, 5)$ and $(17, 6)$.


