% TODO
% . include figures

\section{Faster Candidate Motif Elimination through Block Processing}

The EMS-GT algorithm tests candidate motifs in a brute-force approach. A candidate motif is tested by checking if it has at least one $d$-neighbor in each of the remaining $n - n'$ sequences. In testing a candidate motif $c$, if there is a sequence $S_i$ in the remaining $n - n'$ sequences where $c$ doesn't have any $d$-neighbor, candidate motif $c$ is automatically eliminated.

In our implementation, the search space is represented by a compressed bit array and the $l$-mers are enumerated alphabetically. $L$-mers that are near each other do not differ that much. We used this observation in improving the way the algorithm tests the candidate motifs. Figure 1 illustrates a part of the search space's representation.

% include search space, 
\begin{figure}[h]
	\centering
	\includegraphics[width=5in]{contents/00_images/search_space}\vspace*{5pt}
	\caption{First 8 rows of the $4^{5}$ search space with random flag values.}
	\label{fig:search_space}
\end{figure}

In EMS-GT, each testing of candidate motif is independent of each other. We proposed a speedup technique that processes these candidate motifs by blocks. We first partition the search space by blocks containing $4^k$ $l$-mers. This results into $l$-mers that share the same $(l-k)$-prefix characters where $2 < k < l$. Since every row in the search space represents exactly 32 $l$-mers (32-bit integers), the height of every block is computed using $(4^k / 32)$. Additionally, any two $l$-mers within a block has at most $k$ hamming distance value between them. Figure 2 shows an example of partitioned search space showing blocks of $l$-mers sharing a common prefix string. In our implementation we use $k = 5$ and every block has a height of 32.

% include figure that shows how we partition the search space
\begin{figure}[h]
	\centering
	\label{fig:block_search_space}
	\includegraphics[width=4.5in]{contents/00_images/block_search_space}\vspace*{5pt}
	\caption{Illustration of the block partitioning of a $4^{12}$ search space. There are $4^{5}$ $l$-mers in each block and has a height of 32. The illustration shows the first 3 blocks only.}
\end{figure}

We process the testing of candidate motifs now by blocks. If candidate motifs $x$ and $y$ are within a block and $x$ has been eliminated as a candidate motif in sequence $S_i$ $(n' \leq i \leq n)$, we can filter out $l$-mers $z \in S_i$ where $d_H(x,z) > d + k$. We collect the remaining $l$-mers in $S_i$ and use it for testing the remaining candidate motifs in the block along with the other $l$-mers in the remaining sequences in ${S_n', S_n'+1, ..., S_n} - {S_i}$. The theorem below formalized the main property used in this speedup technique.

\begin{thm} \label{thm:triangle}
	Let x and y be $l$-mers in a block in the search space containing $4^k$ $l$-mers and $d_H(x, y) \leq k$. Let $d$ be the number of allowed mutations in the problem. Let z be another $l$-mer. If $d_H(x, z) > (d + k)$ then $d_H(y, z) > d$ and $z \not\in N(y, d)$
\end{thm}

\begin{proof}[Proof of Theorem \ref{thm:triangle}]
Using proof by contradiction, suppose that $d_H(y, z) \leq d$. We know that $d_H(x, z) < d_H(y, z) + d_H(x, y)$ from the triangle inequality. We can write this equation into $d_H(x, z) < d + k$ thus a contradiction to the previous assumption that $d_H(x, z) > (d + k)$.
\end{proof}

The $k$-value affects the number of $l$-mers that is filtered in a sequence. The lower its value, the larger the number of filtered $l$-mers and faster candidate motif testing will be. But since $k$ also affects the number of $l$-mers in a block, the lower its value, the fewer the candidate motifs that might benefit from the speedup technique.

The Test phase now is executed as follows:

\itemize{}

1. Divide into blocks
2. Will only apply on those blocks with 2 or more candidate motifs
3. Will wait until one candidate motif is eliminated
4. Will filter lmers in the sequence where it got eliminated
5. will use the fileterd sequence instead of the original sequence.