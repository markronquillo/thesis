% Runtime results
\subsection{Performance of EMS-GT with speedup techniques}

The EMS-GT2, EMS-GT and qPMS9 was evaluated using Intel Xeon, 2.10 Ghz machine. The performance of each algorithm was averaged over 20 synthetic datasets for each $(l, d)$-challenge instance where $l \leq 17$. Table \ref{tbl:final-results-ems} shows the runtime results between EMS-GT2 vs EMS-GT while table \ref{tbl:final-results} shows the runtime results between EMS-GT2 vs the state-of-the-art algorithm qPMS9.

\begin{table}[h] %speedup_blockmasking
	\renewcommand{\arraystretch}{1.3}
	\centering
	\begin{tabular}{|c|c|c|c|}
	\hline 
	\bfseries\boldmath $(l,d)$ & 
	\bfseries EMS-GT & 
	\bfseries\boldmath EMS-GT2  & 
	\bfseries \% speedup\\
	\hline
	 (9,2) 	&  0.047 s &    0.038 s &    18.33 \%\\
	(11,3) &   0.168 s &    0.126 s &    25.46 \%\\
	(13,4) &   1.181 s &    0.946 s &   19.88 \%\\
	(15,5) &  12.768 s &   10.876 s &   14.81 \%\\
	(17,6) & 143.215 s &  111.660 s &   22.03 \%\\
	\hline\end{tabular}
	
	\caption{EMS-GT and EMS-GT2 runtime evaluation.}
	\label{tbl:final-results-ems}
\end{table}

% Explanation for Table final-results-ems
The additional speedup techniques become more efficient as the $l$ value in the $(l, d)$-instance grows. For every $(l, d)$-challenge instances mentioned where $l \geq 13$, EMS-GT2 has improved the runtime over the EMS-GT for at least 15\%. Unfortunately, the speedup techniques in EMS-GT2 failed to compensate for their additional overhead computations in both $(9, 2)$ and $(11, 3)$ challenge instances and failed to improve the overall runtime of the implementation.

\begin{table}[h] %speedup_blockmasking
	\renewcommand{\arraystretch}{1.3}
	\centering
	\begin{tabular}{|c|c|c|c|}
	\hline 
	\bfseries\boldmath $(l,d)$ & 
	\bfseries qPMS9 & 
	\bfseries\boldmath EMS-GT2 & 
	\bfseries \% speedup\\
	\hline
	 (9,2) &   0.647 s &    0.039 s &   93.97 \%\\
	(11,3) &   1.276 s &    0.140 s &   89.02 \%\\
	(13,4) &   4.269 s &    0.758 s &   82.24 \%\\
	(15,5) &  24.737 s &   10.565 s &   57.29 \%\\
	(17,6) & 118.226 s &  112.311 s &   5.00 \%\\
	\hline\end{tabular}
	
	\caption{EMS-GT2 and qPMS9 runtime evaluation.}
	\label{tbl:final-results}
\end{table}



% Explanation for Table final-results
Previous implementations of EMS-GT2 failed to beat qPMS9 in $(17, 6)$-challenge instance. EMS-GT2 not only improved the runtime of the implementation but also succeed in beating the qPMS9 in this challenge instance. The implementation of EMS-GT now is faster than the state-of-the-art qPMS9 in all of the $(l, d)$-challenge instances where $l \leq 17$. Even though the implementation of EMS-GT can only run in $(l, d)$-challenge instances where $l \leq 17$ because of memory constraint, studies shown that the typical length of motifs is around 10 base pairs (bp) \cite{stewart2012transcription}





