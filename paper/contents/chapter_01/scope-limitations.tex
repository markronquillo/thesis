\section{Scope and limitations}

This study focuses on improving the existing EMS-GT by exploring and optimizing different areas in the algorithm and thus proposing the EMS-GT2 algorithm. Since EMS-GT2 requires significant amount of memory when $l$ is sufficiently large, each new speedup technique was evaluated using all synthetic datasets of ($l$, $d$)-challenge instance where $l < 18$. Generally, this is acceptable since the motif lengths in actual biological applications is around 10 base pairs (bp) only \cite{stewart2012transcription}. EMS-GT2 is not yet designed to run on multiple processors and, thus, we evaluate the algorithms in a single-processor execution. Finally, we compared EMS-GT2 to its predecessor EMS-GT and the algorithm qPMS9 using a set of challenging ($l$, $d$) instances. An $(l, d)$ problem instance is said to be challenging if $d$ is the largest integer value for which the expected number of motifs of length $l$ would occur in the input by random chance and does not exceed a constant value (500) \cite{pms2015}. The challenge instances used in the evaluation are the following: (9, 2), (11, 3), (13, 4), (15, 5) and (17, 6).

% TODO: cite base pairs